\documentclass[11pt]{article}

\usepackage[margin=1in]{geometry}
\usepackage{amsmath,amsfonts,amssymb}
\usepackage[none]{hyphenat}
\usepackage{fancyhdr}

\usepackage{multicol}
\usepackage{graphicx}
\usepackage{pgfplots}
\usepackage{wrapfig}
\usepackage{gensymb}

\pagestyle{fancy}
\fancyhead{}
\fancyfoot{}
\fancyhead[L]{\slshape \MakeUppercase{1.1 - Four Ways to Represent a Function}}
\fancyhead[R]{\slshape Ho Ngoc Van}
\fancyfoot[C]{\thepage}
\renewcommand{\footrulewidth}{0pt}

\newcommand{\soln}{\subsection*}
\newcommand{\qn}{\textit}
\newcommand{\imagesource}[1]{{\footnotesize Source: #1}}
\newcommand{\img}[1]{
	\begin{figure}[h]
		\centering
		\includegraphics[width=0.35\linewidth]{figs/#1}\\
		\imagesource{James Stewart, Calculus: Early Transcendentals [9e]}
	\end{figure}
}

\begin{document}

\section*{Problem 1}

\qn{If $f(x)=x+\sqrt{2-x}$ and $g(x)=u+\sqrt{2-u}$, is it true that $f=g$?}

\soln{Solution}

True


\section*{Problem 2}

\qn{If $$f(x)=\frac{x^2-x}{x-1} \quad\mathrm{and}\quad g(x)=x$$ is it true that $f=g$?}

\soln{Solution}

False

\section*{Problem 3}

\qn{The graph of a function $g$ is given:}

\img{1.1.3}

\begin{enumerate}
	\item \qn{State the values of $g(-2)$, $g(0)$, $g(2)$ and $g(3)$}
	\soln{Solution}
	$g(-2)=2 \quad g(0)=-2 \quad g(2)=1 \quad g(3)=2.5$
	
	\item \qn{For what value(s) of $x$ is $g(x)=3$?}
	\soln{Solution}
	$g(x)=3 \Rightarrow x=-4$
	
	\item \qn{For what value(s) of $x$ is $g(x) \leq 3$?}
	\soln{Solution}
	$g(x) \leq 3 \Rightarrow x \in [-4,4]$
	
	\item \qn{State the domain and range of $g$}
	\soln{Solution}
	$\mathrm{Domain}:[-4,4] \qquad\mathrm{Range}:[-2,3]$
	
	\item \qn{On what interval(s) is $g$ increasing?}
	\soln{Solution}
	$[0,2]$
\end{enumerate}

\section*{Problem 4}

\qn{The graph of $f$ and $g$ are given:}

\img{1.1.4}

\begin{enumerate}
	\item \qn{State the values of $f(-4)$ and $g(3)$}
	\soln{Solution}
	$f(-4)=-2 \quad g(3)=4$
	
	\item \qn{Which is larger, $f(-3)$ or $g(-3)$?}
	\soln{Solution}
	$g(-3)$
	
	\item \qn{For what values of $x$ is $f(x)=g(x)$?}
	\soln{Solution}
	$x= \pm 2$
	
	\item \qn{On what interval(s) is $f(x) \leq g(x)$?}
	\soln{Solution}
	$[-4,-2] \cup [2,3]$
	
	\item \qn{State the solution of the equation $f(x)=-1$}
	\soln{Solution}
	$f(x)=-1 \Rightarrow x=-3$
	
	\item \qn{On what interval(s) is $g$ decreasing?}
	\soln{Solution}
	$[-4,0]$
	
	\item \qn{State the domain and range of $f$}
	\soln{Solution}
	$\mathrm{Domain}:[-4,4] \qquad\mathrm{Range}:[-2,3]$
	
	\item \qn{State the domain and range of $g$}
	\soln{Solution}
	$\mathrm{Domain}:[-4,3] \qquad\mathrm{Range}:[0.5,4]$
\end{enumerate}

\section*{Problem 5}

\qn{Figure 1 was recorded by an instrument operated by the California Department of Mines and Geology at the University Hospital of the University of Southern California in Los Angeles. Use it to estimate the range of the vertical ground acceleration function at USC during the North-ridge earthquake.}

\soln{Solution}

\section*{Problem 6}

\qn{In this section we discussed examples of ordinary, everyday functions: population is a function of time, postage cost is a function of package weight, water temperature is a function of time. Give three other examples of function from everyday life that are described verbally. What can you say about the domain and range of each of your functions? If possible, sketch a rough graph of each function.}

\soln{Solution}

\section*{Problem 7}

\qn{Determine whether the equation or table defines $y$ as a function of $x$: $$3x-5y=7$$}

\soln{Solution}
True
$$y=\frac{3x-7}{5}$$

\section*{Problem 8}

\qn{Determine whether the equation or table defines $y$ as a function of $x$: $$3x^2-2y=5$$}

\soln{Solution}
True
$$y=\frac{3x^2-5}{2}$$

\section*{Problem 9}

\qn{Determine whether the equation or table defines $y$ as a function of $x$: $$x^2+(y-3)^2=5$$}

\soln{Solution}
False
$$y=\pm(\sqrt{3x^2-5}+3)$$

\section*{Problem 10}

\qn{Determine whether the equation or table defines $y$ as a function of $x$: $$2xy+5y^2=4$$}

\soln{Solution}
False
$$y=\frac{-x\pm\sqrt{x^2+20}}{5}$$

\section*{Problem 11}

\qn{Determine whether the equation or table defines $y$ as a function of $x$: $$(y+3)^3+1=2x$$}

\soln{Solution}
True
$$y=\sqrt[3]{2x-1}-3$$

\section*{Problem 12}

\qn{Determine whether the equation or table defines $y$ as a function of $x$: $$2x-|y|=0$$}

\soln{Solution}
False
$$y=\pm2x$$

\section*{Problem 13}

\qn{Determine whether the equation or table defines $y$ as a function of $x$:}

\begin{tabular}{|c|c|}
	\hline
	x (Height) (in) & y (Shoe size) \\
	\hline
	72 & 12 \\
	\hline
	60 & 8 \\
	\hline
	60 & 7 \\
	\hline
	63 & 9 \\
	\hline
	70 & 10 \\
	\hline
\end{tabular}

\soln{Solution}
False

\section*{Problem 14}

\qn{Determine whether the equation or table defines $y$ as a function of $x$:}

\begin{tabular}{|c|c|}
	\hline
	x (Year) & y (Tuition cost) (\$) \\
	\hline
	2016 & 10,900 \\
	\hline
	2017 & 11,000 \\
	\hline
	2018 & 11,200 \\
	\hline
	2019 & 11,200 \\
	\hline
	2020 & 11,300 \\
	\hline
\end{tabular}

\soln{Solution}
True

\section*{Problem 15}

\qn{Determine whether the curve is the graph of a function of $x$. If it is, state the domain and range of the function}

\img{1.1.15}

\soln{Solution}
False

\section*{Problem 16}

\qn{Determine whether the curve is the graph of a function of $x$. If it is, state the domain and range of the function}

\img{1.1.16}

\soln{Solution}
$\mathrm{Domain}:[-2,2] \qquad\mathrm{Range}:[-1,2]$

\section*{Problem 17}

\qn{Determine whether the curve is the graph of a function of $x$. If it is, state the domain and range of the function}

\img{1.1.17}

\soln{Solution}
$\mathrm{Domain}:[-3,2] \qquad\mathrm{Range}:[-3,3]$

\section*{Problem 18}

\qn{Determine whether the curve is the graph of a function of $x$. If it is, state the domain and range of the function}

\img{1.1.18}

\soln{Solution}
False

\section*{Problem 19}

\qn{Shown is a graph of the global average temperature $T$ during the 20th century. Estimate the following:}

\img{1.1.19}

\begin{enumerate}
	\item \qn{The global average temperature in 1950}
	\soln{Solution}
	$\approx 13.82$
	
	\item \qn{The year when the average temperature was $14.2 \degree C$}
	\soln{Solution}
	$\approx 1992$
	
	\item \qn{The years when the temperature was smallest and largest}
	\soln{Solution}
	$1910 \quad \text{and} \quad 2003$
	
	\item \qn{The range of $T$}
	\soln{Solution}
	$[13.5, 14.4]$
\end{enumerate}

\section*{Problem 20}

\qn{Trees grow faster and form wider rings in warm years and grow more slowly and form narrower rings in cooler years. The figure shows ring widths of a Siberian pine from 1500 to 2000.}

\img{1.1.20}

\begin{enumerate}
	\item \qn{What is the range of the ring width function?}
	\soln{Solution}
	$[0.1, 1.6]$ (mm)
	
	\item \qn{What does the graph tend to say about the temperature of the earth? Does the graph reflect the volcanic eruptions of the mid-19th century?}
	\soln{Solution}
	The graph tends to say that the temperature of the earth is increasing. And it also reflects the volcanic eruptions of the mid-19th century.
\end{enumerate}

\section*{Problem 21}

\qn{You put some ice cubes in a glass, fill the glass with cold water, and then let the glass sit on a table. Describe how the temperature of the water changes as time passes. Then sketch a rough graph of the temperature of the water as a function of the elapsed time.}

\soln{Solution}

\section*{Problem 22}

\qn{You place a frozen pie in an oven and bake it for an hour. Then you take it out and let it cool. Describe how the temperature of the pie changes as time passes. Then sketch a rough graph of the temperature of the pie as a function of time.}

\soln{Solution}

\section*{Problem 23}

\qn{The graph shows the power consumption for a day in September in San Francisco. ($P$ is measured in megawatts; $t$ is measured in hours starting at midnight.)}

\img{1.1.23}

\begin{enumerate}
	\item \qn{What was the power consumption at 6 AM? At 6 PM?}
	\soln{Solution}
	The power consumption at 6 AM is 500 (MW), and at 6 PM is 720 (MW).
	
	\item \qn{When was the power consumption the lowest? When was it the highest? Do these times seem reasonable?}
	\soln{Solution}
	The power consumption is lowest at 3 AM and is highest at midday. And it is reasonable.
\end{enumerate}


\section*{Problem 24}

\qn{Three runners compete in a 100-meter race. The graph depicts the distance run as a function of time for each runner. Describe in words what the graph tells you about this race. Who won the race? Did each runner finish the race?}

\img{1.1.24}

\soln{Solution}

\section*{Problem 25}

\qn{Sketch a rough graph of the outdoor temperature as a function of time during a typical spring day.}

\soln{Solution}

\section*{Problem 26}

\qn{Sketch a rough graph of the number of hours of daylight as a function of the time of year.}

\soln{Solution}

\section*{Problem 27}

\qn{Sketch a rough graph of the amount of a particular brand of coffee sold by a store as a function of the price of the coffee.}

\soln{Solution}

\section*{Problem 28}

\qn{Sketch a rough graph of the market value of a new car as a function of time for a period of 20 years. Assume the car is well maintained.}

\soln{Solution}

\section*{Problem 29}

\qn{A homeowner mows the lawn every Wednesday afternoon. Sketch a rough graph of the height of the grass as a function of time over the course of a four-week period.}

\soln{Solution}

\section*{Problem 30}

\qn{An airplane takes off from an airport and lands an hour later at another airport, 400 miles away. If $t$ represents the time in minutes since the plane has left the terminal building, let $x(t)$ be the horizontal distance traveled and $y(t)$ be the altitude of the plane.}

\begin{enumerate}
	\item \qn{Sketch a possible graph of $x(t)$}
	\soln{Solution}
	
	\item \qn{Sketch a possible graph of $y(t)$}
	\soln{Solution}
	
	\item \qn{Sketch a possible graph of the ground speed}
	\soln{Solution}
	
	\item \qn{Sketch a possible graph of the vertical velocity}
	\soln{Solution}
\end{enumerate}

\end{document}
