\documentclass[11pt]{article}
\usepackage[utf8]{vietnam}

\usepackage[margin=1in]{geometry}
\usepackage{amsmath,amsfonts,amssymb}
\usepackage[none]{hyphenat}
\usepackage{fancyhdr}

\usepackage{multicol}
\usepackage{graphicx}
\usepackage{pgfplots}
\usepackage{wrapfig}
\usepackage{gensymb}
\usepackage{float}

\pagestyle{fancy}
\fancyhead{}
\fancyfoot{}
\fancyhead[L]{\slshape \MakeUppercase{1.3 - New Functions from Old Functions}}
\fancyhead[R]{\slshape Ho Ngoc Van}
\fancyfoot[C]{\thepage}
\renewcommand{\footrulewidth}{0pt}

\newcommand{\soln}{\subsection*}
\newcommand{\qn}{\textit}
\newcommand{\imagesource}[1]{{\footnotesize Source: #1}}
\newcommand{\imgqn}[1]{
	\begin{figure}[H]
		\centering
		\includegraphics[width=0.35\linewidth]{figs/#1.png}\\
		\imagesource{James Stewart, Calculus: Early Transcendentals [9e]}
	\end{figure}
}
\newcommand{\imgsoln}[1]{
	\begin{figure}[H]
		\centering
		\includegraphics[width=0.5\linewidth]{figs/#1.png}
	\end{figure}
}

\begin{document}

\section*{Problem 1}

\qn{Support the graph of $f$ is given. Write equations for the graphs that are obtained from the graph of $f$ as follows}

\begin{enumerate}
	\item \qn{Shift 3 units upward}
	\soln{Solution}
	$f(x)+3$
	
	\item \qn{Shift 3 units downward}
	\soln{Solution}
	$f(x)-3$
	
	\item \qn{Shift 3 units to the right}
	\soln{Solution}
	$f(x-3)$
	
	\item \qn{Shift 3 units to the left}
	\soln{Solution}
	$f(x+3)$
	
	\item \qn{Reflect about the $x$-axis}
	\soln{Solution}
	$-f(x)$
	
	\item \qn{Reflect about the $y$-axis}
	\soln{Solution}
	$f(-x)$
	
	\item \qn{Stretch vertically by a factor of 3}
	\soln{Solution}
	$3f(x)$
	
	\item \qn{Shrink vertically by a factor of 3}
	\soln{Solution}
	$\frac{1}{3}f(x)$
\end{enumerate}

\section*{Problem 2}

\qn{Explain how each graph is obtained from the graph of $y=f(x)$}

\begin{enumerate}
	\item \qn{$y=f(x)+8$}
	\soln{Solution}
	Dịch lên trên 8 đơn vị
	
	\item \qn{$y=f(x+8)$}
	\soln{Solution}vị
	Dịch bên trái 8 đơn vị
	
	\item \qn{$y=8f(x)$}
	\soln{Solution}
	Kéo dài theo chiều dọc thêm 8 lần
	
	\item \qn{$y=f(8x)$}
	\soln{Solution}
	Co lại theo chiều ngang 8 lần
	
	\item \qn{$y=-f(x)-1$}
	\soln{Solution}
	Lật đối xứng theo trục $x$ xong sau đó dịch xuống dưới 1 đơn vị
	
	\item \qn{$y=8f(\frac{1}{8}x)$}
	\soln{Solution}
	Mở rộng theo chiều ngang 8 lần sau đó kéo dài theo chiều dọc 8 lần
\end{enumerate}

\section*{Problem 3}

\qn{The graph of $y=f(x)$ is given. Match each equation with its graph and give reasons for your choices}
\imgqn{1.3.3}

\begin{enumerate}
	\item \qn{$y=f(x-4)$}
	\soln{Solution}
	Dịch sang phải 4 đơn vị $\Rightarrow$ (3)
	
	\item \qn{$y=f(x)+3$}
	\soln{Solution}
	Dịch lên trên 3 đơn vị $\Rightarrow$ (1)
	
	\item \qn{$y=\frac{1}{3}f(x)$}
	\soln{Solution}1
	Co lại theo chiều dọc 3 lần $\Rightarrow$ (4)
	
	\item \qn{$y=-f(x+4)$}
	\soln{Solution}
	Dịch trái 4 đơn vị sau đó lật theo trục $x$ $\Rightarrow$ (5)
	
	\item \qn{$y=2f(x+6)$}
	\soln{Solution}
	Dịch trái 6 đơn vị sau đó kéo dài theo chiều dọc 2 lần $\Rightarrow$ (2)
\end{enumerate}

\section*{Problem 4}

\qn{The graph of $f$ is given. Draw the graphs of the following functions}
\imgqn{1.3.4}

\begin{enumerate}
	\item \qn{$y=f(x)-3$}
	\soln{Solution}
	Dịch xuống dưới 3 đơn vị
	
	\item \qn{$y=f(x+1)$}
	\soln{Solution}vị
	Dịch sang trái 1 đơn vị
	
	\item \qn{$y=\frac{1}{2}f(x)$}
	\soln{Solution}
	Co lại theo chiều dọc 2 lần
	
	\item \qn{$y=-f(x)$}
	\soln{Solution}
	Lật đối xứng qua trục $x$
\end{enumerate}

\section*{Problem 5}

\qn{The graph of $f$ is given. Use it to graph the following functions}
\imgqn{1.3.5}

\begin{enumerate}
	\item \qn{$y=f(2x)$}
	\soln{Solution}
	Co lại theo chiều ngang 2 lần
	
	\item \qn{$y=f(\frac{1}{2}x)$}
	\soln{Solution}
	Dãn ra theo chiều ngang 2 lần
	
	\item \qn{$y=f(-x)$}
	\soln{Solution}
	Lật đối xứng qua trục $y$
	
	\item \qn{$y=-f(-x)$}
	\soln{Solution}
	Trước tiên là lật đối xứng qua trục $x$ sau đó lật đối xứng qua trục $y$
\end{enumerate}

\section*{Problem 6}

\qn{The graph of $y=\sqrt{3x-x^2}$ is given. Use transformations to create a function whose graph is as shown}
\imgqn{1.3.6+7}

\soln{Solution}
\imgqn{1.3.6}
Đồ thị bị dịch sang phải 2 đơn vị, sau đó kéo dài theo chiều dọc 2 lần $$g(x)=2f(x-2)=2\sqrt{3(x-2)-(x-2)^2}=2\sqrt{-x^2+7x-10}$$

\section*{Problem 7}

\qn{The graph of $y=\sqrt{3x-x^2}$ is given. Use transformations to create a function whose graph is as shown}
\imgqn{1.3.6+7}

\soln{Solution}
\imgqn{1.3.7}
Đồ thị dịch trái 4 đơn vị, sau đó lật đối xứng qua trục $x$, cuối cùng là dịch xuống dưới 1 đơn vị $$g(x)=-f(x+4)-1=-\sqrt{3(x+4)-(x+4)^2}-1=-\sqrt{-x^2-5x-4}-1$$

\section*{Problem 8}

\begin{enumerate}
	\item \qn{How is graph of $y=1+\sqrt{x}$ related to the graph of $y=\sqrt{x}$? Use your answer to sketch the graph of $y=1+\sqrt{x}$}
	\soln{Solution}
	Đồ thị hàm $y=1+\sqrt{x}$ được vẽ bằng cách dịch đồ thị hàm số $y=\sqrt{x}$ lên trên 1 đơn vị
	\imgsoln{1.3.8-ans.a}
	
	\item \qn{How is the graph of $y=5\sin{\pi x}$ related to the graph of $y=\sin{x}$? Use your answer to sketch the graph of $y=5\sin{\pi x}$}
	\soln{Solution}
	Đồ thị hàm số $y=5\sin{\pi x}$ được vẽ bằng cách co đồ thị hàm số $y=\sin{x}$ theo chiều ngang lại $\pi$ lần, sau đó kéo dài theo trục $y$ 5 lần
	\imgsoln{1.3.8-ans.b}
\end{enumerate}

\section*{Problem 9}

\qn{Graph the function by hand, not by plotting points, but by starting with the graph of one of the standard functions, and then applying the appropriate transformations $$y=1+x^2$$}

\soln{Solution}
Đồ thị được vẽ bằng cách dịch đồ thị hàm số $y=x^2$ lên 1 đơn vị

\section*{Problem 10}

\qn{Graph the function by hand, not by plotting points, but by starting with the graph of one of the standard functions, and then applying the appropriate transformations $$y=(x+1)^2$$}

\soln{Solution}
Đồ thị được vẽ bằng cách dịch đồ thị hàm số $y=x^2$ sang trái 1 đơn vị

\section*{Problem 11}

\qn{Graph the function by hand, not by plotting points, but by starting with the graph of one of the standard functions, and then applying the appropriate transformations $$y=|x+2|$$}

\soln{Solution}vị
Đồ thị được vẽ bằng cách dịch đồ thị hàm số $y=|x|$ sang trái 2 đơn vị

\section*{Problem 12}

\qn{Graph the function by hand, not by plotting points, but by starting with the graph of one of the standard functions, and then applying the appropriate transformations $$y=1-x^3$$}

\soln{Solution}
Đồ thị được vẽ bằng cách lấy đối xứng qua trục $x$ đồ thị hàm số $y=x^3$ sau đó dịch đồ thị thu được lên 1 đơn vị

\section*{Problem 13}

\qn{Graph the function by hand, not by plotting points, but by starting with the graph of one of the standard functions, and then applying the appropriate transformations $$y=\frac{1}{x}+2$$}

\soln{Solution}
Đồ thị được vẽ bằng cách dịch đồ thị hàm số $y=\frac{1}{x}$ lên 2 đơn vị

\section*{Problem 14}

\qn{Graph the function by hand, not by plotting points, but by starting with the graph of one of the standard functions, and then applying the appropriate transformations $$y=-\sqrt{x}-1$$}

\soln{Solution}
Đồ thị được vẽ bằng cách lấy đối xứng qua trục $x$ đồ thị hàm số $y=\sqrt{x}$ sau đó dịch đồ thị thu được xuống 1 đơn vị

\section*{Problem 15}

\qn{Graph the function by hand, not by plotting points, but by starting with the graph of one of the standard functions, and then applying the appropriate transformations $$y=\sin{4x}$$}

\soln{Solution}
Đồ thị được vẽ bằng cách co đồ thị hàm số $y=\sin{x}$ vào 1/4 lần theo chiều ngang

\section*{Problem 16}

\qn{Graph the function by hand, not by plotting points, but by starting with the graph of one of the standard functions, and then applying the appropriate transformations $$y=1+\frac{1}{x^2}$$}

\soln{Solution}
Đồ thị được vẽ bằng cách dịch đồ thị hàm số $y=\frac{1}{x^2}$ lên 1 đơn vị

\section*{Problem 17}

\qn{Graph the function by hand, not by plotting points, but by starting with the graph of one of the standard functions, and then applying the appropriate transformations $$y=2+\sqrt{x+1}$$}

\soln{Solution}
Đồ thị được vẽ bằng cách dịch đồ thị hàm số $y=\sqrt{x}$ sang trái 1 đơn vị, sau đó dịch đồ thị hàm số thu được lên trên 2 đơn vị

\section*{Problem 18}

\qn{Graph the function by hand, not by plotting points, but by starting with the graph of one of the standard functions, and then applying the appropriate transformations $$y=-(x-1)^2+3$$}

\soln{Solution}
Đồ thị được vẽ bằng cách dịch đồ thị hàm số $y=x^2$ sang phải 1 đơn vị, lấy đối xứng qua trục $x$ đồ thị đó và cuối cùng dịch đồ thị thu được lên trên 3 đơn vị

\section*{Problem 19}

\qn{Graph the function by hand, not by plotting points, but by starting with the graph of one of the standard functions, and then applying the appropriate transformations $$y=x^2-2x+5$$}

\soln{Solution}
Đồ thị $y=x^2-2x+5=(x-1)^2+4$ được vẽ bằng cách dịch đồ thị hàm số $y=x^2$ sang phải 1 đơn vị sau đó dịch đồ thị thu được lên trên 4 đơn vị

\section*{Problem 20}

\qn{Graph the function by hand, not by plotting points, but by starting with the graph of one of the standard functions, and then applying the appropriate transformations $$y=(x+1)^3+2$$}

\soln{Solution}
Đồ thị được vẽ bằng cách dịch đồ thị hàm số $y=x^3$ sang trái 1 đơn vị, sau đó dịch đồ thị vừa thu được lên trên 2 đơn vị


\section*{Problem 21}

\qn{Graph the function by hand, not by plotting points, but by starting with the graph of one of the standard functions, and then applying the appropriate transformations $$y=2-|x|$$}

\soln{Solution}
Đồ thị được vẽ bằng cách lấy đối xứng qua trục $x$ đồ thị hàm số $y=|x|$, sau đó dịch đồ thị vừa thu được lên trên 2 đơn vị

\section*{Problem 22}

\qn{Graph the function by hand, not by plotting points, but by starting with the graph of one of the standard functions, and then applying the appropriate transformations $$y=2-2\cos{x}$$}

\soln{Solution}
Đồ thị được vẽ bằng cách lấy đối xứng qua trục $x$ đồ thị hàm số $y=\cos{x}$, sau đó kéo đồ thị vừa thu được lên gấp 2 lần theo chiều dọc, cuối cùng dịch đồ thị đó lên trên 2 đơn vị

\section*{Problem 23}

\qn{Graph the function by hand, not by plotting points, but by starting with the graph of one of the standard functions, and then applying the appropriate transformations $$y=3\sin{\frac{1}{2}x}+1$$}

\soln{Solution}
Đồ thị được vẽ bằng dãn đồ thị hàm số $y=\sin{x}$ theo chiều ngang 2 lần, kéo đồ thị thu được theo chiều dọc 3 lần, và cuối cùng dịch đồ thị đó lên trên 1 đơn vị

\section*{Problem 24}

\qn{Graph the function by hand, not by plotting points, but by starting with the graph of one of the standard functions, and then applying the appropriate transformations $$y=\frac{1}{4}\tan(x-\frac{\pi}{4})$$}

\soln{Solution}
Đồ thị được vẽ bằng cách dịch đồ thị hàm số $y=\tan{x}$ sang phải $\frac{\pi}{4}$ đơn vị sau đó co đồ thị theo chiều dọc lại 4 lần

\section*{Problem 25}

\qn{Graph the function by hand, not by plotting points, but by starting with the graph of one of the standard functions, and then applying the appropriate transformations $$y=|\cos(\pi x)|$$}

\soln{Solution}
Đồ thị được vẽ bằng cách co đồ thị $y=\cos{x}$ lại theo chiều dọc $\pi$ lần, sau đó lấy đối xứng qua trục $x$ phần đồ thị có giá trị âm


\section*{Problem 26}

\qn{Graph the function by hand, not by plotting points, but by starting with the graph of one of the standard functions, and then applying the appropriate transformations $$y=|\sqrt{x}-1|$$}

\soln{Solution}
Đồ thị được vẽ bằng cách dịch đồ thị hàm số $y=\sqrt{x}$ xuống dưới 1 đơn vị, sau đó lấy đối xứng qua trục $x$ của phần đồ thị có giá trị âm

\section*{Problem 27}

\qn{The city of New Orleans is located at latitude $30\degree N$. Use figure to find a function that models the number of hours of day-light at New Orleans as a function of the time of year. To check the accuracy of your model, use the fact that on March 31 the sun rises at 5:51 AM and sets at 6:18 PM in New Orleans}
\imgqn{1.3.27}

\soln{Solution}

\section*{Problem 28}

\qn{A variable star is one whose brightness alternately increases and decreases. For the most visible variable star, Delta Cephei, the time between periods of maximum brightness is 5.4 days, the average brightness (or magnitude) of the star is 4.0, and its brightness varies by $\pm$0.35 magnitude. Find a function that models the brightness of Delta Cephei as a function of time}

\soln{Solution}

\section*{Problem 29}

\qn{Some of the highest tides in the world occur in the Bay of Fundy on the Atlantic Coast of Canada. At Hopewell Cape the water depth at low tide is about 2.0 m and at high tide it is about 12.0 m. The natural period of oscillation is about 12 hours and on a particular day, high tide occurred at 6:45 am. Find a function involving the cosine function that models the water depth $D(t)$ (in meters) as a function of time $t$ (in hours after midnight) on that day}

\soln{Solution}

\section*{Problem 30}

\qn{In a normal respiratory cycle the volume of air that moves into and out of the lungs is about 500 ml. The reserve and residue volumes of air that remain in the lungs occupy about 2000 mL and a single respiratory cycle for an average human takes about 4 seconds. Find a model for the total volume of air $V(t)$ in the lungs as a function of time}

\soln{Solution}

\section*{Problem 31}

\begin{enumerate}
	\item \qn{How is the graph of $y=f(|x|)$ related to graph of $f$?}
	\soln{Solution}
	\begin{equation*}
		f(|x|)=
		\begin{cases}
			f(x) \quad \text{if} \quad x \ge 0 \\
			f(-x) \quad \text{if} \quad x < 0
		\end{cases}
	\end{equation*}
	Do $x \in f(x)$ thì $-x \in f(x)$ nên đồ thị hàm số được vẽ bằng cách bỏ đi phần đồ thị có giá trị $x < 0$ và lấy đối xứng qua trục $y$ của phần đồ thị $y=f(x)$ với $x \ge 0$
	
	\item \qn{Sketch the graph of $y=\sin|x|$}
	\soln{Solution}
	\imgsoln{1.3.31-ans.b}
	
	\item \qn{Sketch the graph of $y=\sqrt{|x|}$}
	\soln{Solution}
	\imgsoln{1.3.31-ans.c}
\end{enumerate}

\section*{Problem 32}

\qn{Use the given graph of $f$ to sketch the graph of $y=\frac{1}{f(x)}$. Which features of $f$ are most important is sketching $y=\frac{1}{f(x)}$? Explain how they are used}
\imgqn{1.3.32}

\soln{Solution}
\begin{enumerate}
	\item Khi giá trị $f(x)$ càng lớn thì giá trị hàm số $g(x)=\frac{1}{f(x)}$ càng nhỏ và với những khoảng mà giá trị $f(x)$ càng nhỏ, đồ thị hàm số $g(x)=\frac{1}{f(x)}$ có xu hướng tiến về vô cực.
	
	\item Với hai giá trị $x=\pm 1$ thì $g(x)$ không tồn tại do $f(x)=0$
	
	\item Tại $x=0$ thì $f(x)=g(x)=1$
\end{enumerate}

\section*{Problem 33}

\qn{Find $f+g$, $f-g$, $fg$, $f/g$ and state their domains $$f(x)=\sqrt{25-x^2} \quad \text{and} \quad g(x)=\sqrt{x+1}$$}

\soln{Solution}
Domain của $f$: $[-5,5]$ và domain của $g$: $[-1, \infty]$
\begin{enumerate}
	\item Domain của $f+g$: $[-1,5]$, phép giao của hai domain $$f+g=\sqrt{25-x^2}+\sqrt{x+1}$$
	
	\item Domain của $f-g$: $[-1,5]$, phép giao của hai domain $$f-g=\sqrt{25-x^2}-\sqrt{x+1}$$
	
	\item Domain của $fg$: $[-1,5]$, phép giao của hai domain $$fg=\sqrt{25-x^2}\sqrt{x+1}$$
	
	\item Domain của $f/g$: $(-1,5]$, phép giao của hai domain và $g(x) \ne 0$ $$f/g=\frac{\sqrt{25-x^2}}{\sqrt{x+1}}$$
\end{enumerate}

\section*{Problem 34}

\qn{Find $f+g$, $f-g$, $fg$, $f/g$ and state their domains $$f(x)=\frac{1}{x-1} \quad \text{and} \quad g(x)=\frac{1}{x}-2$$}

\soln{Solution}
Domain của $f$: $(-\infty, 1) \cup (1, \infty)$ và domain của $g$: $(-\infty, 0) \cup (0, \infty)$
\begin{enumerate}
	\item Domain của $f+g$: $x \ne 0, 1$, phép giao của hai domain $$f+g=\frac{1}{x-1}+\frac{1}{x}-2$$
	
	\item Domain của $f-g$: $x \ne 0, 1$, phép giao của hai domain $$f-g=\frac{1}{x-1}-\frac{1}{x}+2$$
	
	\item Domain của $fg$: $x \ne 0, 1$, phép giao của hai domain $$fg=\frac{1}{x-1}(\frac{1}{x}-2)=\frac{1-2x}{x(x-1)}$$
	
	\item Domain của $f/g$: $x \ne 0, 1, 2$, phép giao của hai domain, và $g(x) \ne 0$ $$f/g=\frac{\frac{1}{x-1}}{\frac{1}{x}-2}=\frac{x}{(x-1)(1-2x)}$$
\end{enumerate}

\section*{Problem 35}

\qn{Find the function $f \circ g$, $g \circ f$, $f \circ f$, $g \circ g$ and their domains $$f(x)=x^3+5 \quad \text{and} \quad g(x)=\sqrt[3]{x}$$}

\soln{Solution}
Domain của $f$: $(-\infty, \infty)$ và domain của $g$: $(-\infty, \infty)$
\begin{enumerate}
	\item Domain của $f \circ g$: $(-\infty, \infty)$ $$f \circ g=(\sqrt{x})^3+5=x+5$$
	
	\item Domain của $g \circ f$: $(-\infty, \infty)$ $$g \circ f=\sqrt{x^3+5}$$
	
	\item Domain của $f \circ f$: $(-\infty, \infty)$ $$f \circ f=(x^3+5)^3+5=x^9+15x^6+75x^3+130$$
	
	\item Domain của $g \circ g$: $(-\infty, \infty)$ $$g \circ g=\sqrt[3]{\sqrt[3]{x}}=\sqrt[9]{x}$$
\end{enumerate}

\section*{Problem 36}

\qn{Find the function $f \circ g$, $g \circ f$, $f \circ f$, $g \circ g$ and their domains $$f(x)=\frac{1}{x} \quad \text{and} \quad g(x)=2x+1$$}

\soln{Solution}
Domain của $f$: $(-\infty, 0) \cup (0, \infty)$ và domain của $g$: $(-\infty, \infty)$
\begin{enumerate}
	\item Domain của $f \circ g$: $g(x) \ne 0 \Rightarrow x \ne -1/2$ $$f \circ g=\frac{1}{2x+1}$$
	
	\item Domain của $g \circ f$: $x \ne 0$ $$g \circ f=\frac{2}{x}+1$$
	
	\item Domain của $f \circ f$: $x \ne 0$ $$f \circ f=\frac{1}{\frac{1}{x}}=x$$
	
	\item Domain của $g \circ g$: $(-\infty, \infty)$ $$g \circ g=2(2x+1)+1=4x+3$$
\end{enumerate}

\section*{Problem 37}

\qn{Find the function $f \circ g$, $g \circ f$, $f \circ f$, $g \circ g$ and their domains $$f(x)=\frac{1}{\sqrt{x}} \quad \text{and} \quad g(x)=x+1$$}

\soln{Solution}
Domain của $f$: $x>0$ và domain của $g$: $(-\infty, \infty)$
\begin{enumerate}
	\item Domain của $f \circ g$: $x>0$ $$f \circ g=\frac{1}{\sqrt{x+1}}$$
	
	\item Domain của $g \circ f$: $x>0$ $$g \circ f=\frac{1}{\sqrt{x}}+1$$
	
	\item Domain của $f \circ f$: $x>0$ $$f \circ f=\frac{1}{\sqrt{\frac{1}{\sqrt{x}}}}=\sqrt[4]{x}$$
	
	\item Domain của $g \circ g$: $(-\infty, \infty)$ $$g \circ g=(x+1)+1=x+2$$
\end{enumerate}

\section*{Problem 38}

\qn{Find the function $f \circ g$, $g \circ f$, $f \circ f$, $g \circ g$ and their domains $$f(x)=\frac{x}{x+1} \quad \text{and} \quad g(x)=2x-1$$}

\soln{Solution}
Domain của $f$: $x \ne -1$ và domain của $g$: $(-\infty, \infty)$
\begin{enumerate}
	\item Domain của $f \circ g$: $2x-1 \ne -1 \Rightarrow x \ne 0$ $$f \circ g=\frac{2x-1}{(2x-1)+1}=\frac{2x-1}{2x}$$
	
	\item Domain của $g \circ f$: $x \ne -1$ $$g \circ f=2\frac{x}{x+1}-1=\frac{x-1}{x+1}$$
	
	\item Domain của $f \circ f$: $x \ne -1, -1/x$ $$f \circ f=\frac{\frac{x}{x+1}}{\frac{x}{x+1}+1}=\frac{x}{2x+1}$$
	
	\item Domain của $g \circ g$: $(-\infty, \infty)$ $$g \circ g=2(2x-1)-1=4x-3$$
\end{enumerate}

\section*{Problem 39}

\qn{Find the function $f \circ g$, $g \circ f$, $f \circ f$, $g \circ g$ and their domains $$f(x)=\frac{2}{x} \quad \text{and} \quad g(x)=\sin{x}$$}

\soln{Solution}
Domain của $f$: $x \ne 0$ và domain của $g$: $(-\infty, \infty)$
\begin{enumerate}
	\item Domain của $f \circ g$: $\sin{x} \ne 0 \Rightarrow x \ne k\pi $ $$f \circ g=\frac{2}{\sin{x}}$$
	
	\item Domain của $g \circ f$: $x \ne 0$ $$g \circ f=\sin{\frac{2}{x}}$$
	
	\item Domain của $f \circ f$: $x \ne 0$ $$f \circ f=\frac{2}{\frac{2}{x}}=x$$
	
	\item Domain của $g \circ g$: $(-\infty, \infty)$ $$g \circ g=\sin(\sin{x})$$
\end{enumerate}

\section*{Problem 40}

\qn{Find the function $f \circ g$, $g \circ f$, $f \circ f$, $g \circ g$ and their domains $$f(x)=\sqrt{5-x} \quad \text{and} \quad g(x)=\sqrt{x-1}$$}

\soln{Solution}
Domain của $f$: $(-\infty, 5]$ và domain của $g$: $[1, \infty)$
\begin{enumerate}
	\item Domain của $f \circ g$: $[1, 26]$ $$f \circ g=\sqrt{5-\sqrt{x-1}}$$
	
	\item Domain của $g \circ f$: $(-\infty, 4]$ $$g \circ f=\sqrt{\sqrt{5-x}-1}$$
	
	\item Domain của $f \circ f$: $[-20, 5]$ $$f \circ f=\sqrt{5-\sqrt{5-x}}$$
	
	\item Domain của $g \circ g$: $[2, \infty)$ $$g \circ g=\sqrt{\sqrt{x-1}-1}$$
\end{enumerate}

\section*{Problem 41}

\qn{Find $f \circ g \circ h$ $$f(x)=3x-2 \quad \text{and} \quad g(x)=\sin{x} \quad \text{and} \quad h(x)=x^2$$}

\soln{Solution}
$$(f \circ g \circ h)(x)=3\sin(x^2)-2$$

\section*{Problem 42}

\qn{Find $f \circ g \circ h$ $$f(x)=|x-4| \quad \text{and} \quad g(x)=2^x \quad \text{and} \quad h(x)=\sqrt{x}$$}

\soln{Solution}
$$(f \circ g \circ h)(x)=|2^{\sqrt{x}}-4|$$

\section*{Problem 43}

\qn{Find $f \circ g \circ h$ $$f(x)=\sqrt{x-3} \quad \text{and} \quad g(x)=x^2 \quad \text{and} \quad h(x)=x^3+2$$}

\soln{Solution}
$$(f \circ g \circ h)(x)=\sqrt{(x^3+2)^2-3}=\sqrt{x^6+4x^3+1}$$

\section*{Problem 44}

\qn{Find $f \circ g \circ h$ $$f(x)=\tan{x} \quad \text{and} \quad g(x)=\frac{x}{x-1} \quad \text{and} \quad h(x)=\sqrt[3]{x}$$}

\soln{Solution}
$$(f \circ g \circ h)(x)=\tan(\frac{\sqrt[3]{x}}{\sqrt[3]{x}-1})$$

\section*{Problem 45}

\qn{Express the function in the form $f \circ g$ $$F(x)=(2x+x^2)^4$$}

\soln{Solution}
\begin{equation*}
	F(x)=(2x+x^2)^4
	\Rightarrow
	\begin{cases}
		f(x) = x^4\\
		g(x) = 2x+x^2
	\end{cases}
\end{equation*}

\section*{Problem 46}

\qn{Express the function in the form $f \circ g$ $$F(x)=\cos^2{x}$$}

\soln{Solution}
\begin{equation*}
	F(x)=\cos^2{x}
	\Rightarrow
	\begin{cases}
		f(x) = x^2\\
		g(x) = \cos{x}
	\end{cases}
\end{equation*}

\section*{Problem 47}

\qn{Express the function in the form $f \circ g$ $$F(x)=\frac{\sqrt[3]{x}}{\sqrt[3]{x}+1}$$}

\soln{Solution}
\begin{equation*}
	F(x)=\frac{\sqrt[3]{x}}{\sqrt[3]{x}+1}
	\Rightarrow
	\begin{cases}
		f(x) = \frac{x}{x+1}\\
		g(x) = \sqrt[3]{x}
	\end{cases}
\end{equation*}

\section*{Problem 48}

\qn{Express the function in the form $f \circ g$ $$G(x)=\sqrt[3]{\frac{x}{1+x}}$$}

\soln{Solution}
\begin{equation*}
	G(x)=\sqrt[3]{\frac{x}{1+x}}
	\Rightarrow
	\begin{cases}
		f(x) = \sqrt[3]{x}\\
		g(x) = \frac{x}{1+x}
	\end{cases}
\end{equation*}

\section*{Problem 49}

\qn{Express the function in the form $f \circ g$ $$v(t)=\sec(t^2)\tan(t^2)$$}

\soln{Solution}
\begin{equation*}
	v(t)=\sec(t^2)\tan(t^2)
	\Rightarrow
	\begin{cases}
		f(t) = \sec(t)\tan(t)\\
		g(t) = t^2
	\end{cases}
\end{equation*}

\section*{Problem 50}

\qn{Express the function in the form $f \circ g$ $$H(x)=\sqrt{1+\sqrt{x}}$$}

\soln{Solution}
\begin{equation*}
	H(x)=\sqrt{1+\sqrt{x}}
	\Rightarrow
	\begin{cases}
		f(x) = \sqrt{1+x} \\
		g(x) = \sqrt{x}
	\end{cases}
\end{equation*}

\section*{Problem 51}

\qn{Express the function in form $f \circ g \circ h$ $$R(x)=\sqrt{\sqrt{x}-1}$$}

\soln{Solution}
\begin{equation*}
	R(x)=(f \circ g \circ h)(x)=\sqrt{\sqrt{x}-1}
	\Rightarrow
	\begin{cases}
		f(x) = \sqrt{x} \\
		g(x) = x-1 \\
		h(x) = \sqrt{x}
	\end{cases}
\end{equation*}

\section*{Problem 52}

\qn{Express the function in form $f \circ g \circ h$ $$H(x)=\sqrt[8]{2+|x|}$$}

\soln{Solution}
\begin{equation*}
	H(x)=(f \circ g \circ h)(x)=\sqrt[8]{2+|x|}
	\Rightarrow
	\begin{cases}
		f(x) = \sqrt[8]{x} \\
		g(x) = 2+x \\
		h(x) = |x|
	\end{cases}
\end{equation*}

\section*{Problem 53}

\qn{Express the function in form $f \circ g \circ h$ $$S(t)=\sin^2(\cos{t})$$}

\soln{Solution}
\begin{equation*}
	S(t)=(f \circ g \circ h)(t)=\sin^2(\cos{t})
	\Rightarrow
	\begin{cases}
		f(t) = t^2 \\
		g(t) = \sin(t) \\
		h(t) = \cos(t)
	\end{cases}
\end{equation*}

\section*{Problem 54}

\qn{Express the function in form $f \circ g \circ h$ $$H(t)=\cos(\sqrt{\tan{t}}+1)$$}

\soln{Solution}
\begin{equation*}
	S(t)=(f \circ g \circ h)(t)=\sin^2(\cos{t})
	\Rightarrow
	\begin{cases}
		f(t) = \cos(t) \\
		g(t) = \sqrt{t}+1 \\
		h(t) = \tan(t)
	\end{cases}
\end{equation*}

\section*{Problem 55}

\qn{Use the table to evaluate each expression}
\imgqn{1.3.55+56}

\begin{enumerate}
	\item \qn{$f(g(3))$}
	\soln{Solution}
	$f(g(3))=f(4)=6$
	
	\item \qn{$g(f(2))$}
	\soln{Solution}
	$g(f(2))=g(1)=5$
	
	\item \qn{$(f \circ g)(5)$}
	\soln{Solution}
	$(f \circ g)(5)=f(g(5))=f(3)=5$
	
	\item \qn{$(g \circ f)(5)$}
	\soln{Solution}
	$(g \circ f)(5)=g(f(5))=g(2)=3$
\end{enumerate}

\section*{Problem 56}

\qn{Use the table to evaluate each expression}
\imgqn{1.3.55+56}

\begin{enumerate}
	\item \qn{$g(g(g(2)))$}
	\soln{Solution}
	$g(g(g(2)))=g(g(3))=g(4)=1$
	
	\item \qn{$(f \circ f \circ f)(1)$}
	\soln{Solution}
	$(f \circ f \circ f)(1)=f(f(f(1)))=f(f(3))=f(5)=2$
	
	\item \qn{$(f \circ f \circ g)(1)$}
	\soln{Solution}
	$(f \circ f \circ g)(1)=f(f(g(1)))=f(f(5))=f(2)=1$
	
	\item \qn{$(g \circ f \circ g)(3)$}
	\soln{Solution}
	$(g \circ f \circ g)(3)=g(f(g(3)))=g(f(4))=g(6)=2$
\end{enumerate}

\section*{Problem 57}

\qn{Use the given graphs of $f$ and $g$ to evaluate each expression, or explain why it is undefined}
\imgqn{1.3.57}

\begin{enumerate}
	\item \qn{$f(g(2))$}
	\soln{Solution}
	$f(g(2))=f(5)=4$
	
	\item \qn{$g(f(0))$}
	\soln{Solution}
	$g(f(0))=g(0)=3$
	
	\item \qn{$(f \circ g)(0)$}
	\soln{Solution}
	$(f \circ g)(0)=f(g(0))=f(3)=0$
	
	\item \qn{$(g \circ f)(6)$}
	\soln{Solution}
	$(g \circ f)(6)=g(f(6))=g(6)=undefined$
	
	\item \qn{$(g \circ g)(-2)$}
	\soln{Solution}
	$(g \circ g)(-2)=g(g(-2))=g(1)=4$
	
	\item \qn{$(f \circ f)(4)$}
	\soln{Solution}
	$(f \circ f)(4)=f(f(4))=f(2)=-2$
\end{enumerate}

\section*{Problem 58}

\qn{Use the given graphs of $f$ and $g$ to estimate the value of $f(g(x))$ for $x=-5,-4,-3,...,4,5$. Use these estimates to sketch a rough graph of $f \circ g$}
\imgqn{1.3.58}

\soln{Solution}
${-4, -3.3, -2, -0.2, -0.1, -0.2, -2, -3.3, -4, -2, 1.8}$

\section*{Problem 59}

\qn{A stone is dropped into a lake, creating a circular ripple that travels outward at a speed of 60 cm/s.}

\begin{enumerate}
	\item \qn{Express the radius $r$ of this circle as a function of the time $t$ (in seconds)}
	\soln{Solution}
	
	\item \qn{If $A$ is the area of this circle as a function of the radius, find $A \circ r$ and interpret it}
	\soln{Solution}
\end{enumerate}

\section*{Problem 60}

\qn{A spherical balloon is being inflated and the radius of the balloon is increasing at a rate of 2 cm/s.}

\begin{enumerate}
	\item \qn{Express the radius $r$ of the balloon as a function of the time $t$ (in seconds)}
	\soln{Solution}
	
	\item \qn{If $V$ is the volume of the balloon as a function of the radius, find $V \circ r$ and interpret it}
	\soln{Solution}
\end{enumerate}

\section*{Problem 61}

\qn{A ship is moving at a speed of 30 km/h parallel to a straight shoreline. The ship is 6 km from shore and it passes a lighthouse at noon }

\begin{enumerate}
	\item \qn{Express the distance $s$ between the lighthouse and the ship as a function of $d$, the distance the ship has traveled since noon; that is, find $f$ so that $s=f(d)$}
	\soln{Solution}
	
	\item \qn{Express $d$ as a function of $t$, the time elapsed since noon; that is, find $g$ so that $d=g(t)$}
	\soln{Solution}
	
	\item \qn{Find $f \circ g$. What does this function represent?}
	\soln{Solution}
\end{enumerate}

\section*{Problem 62}

\qn{An airplane is flying at a speed of 350 mi/h at an altitude of one mile and passes directly over a radar station at time $t=0$.}

\begin{enumerate}
	\item \qn{Express the horizontal distance $d$ (in miles) that the plane has flown as a function of $t$}
	\soln{Solution}
	
	\item \qn{Express the distance $s$ between the plane and the radar station as a function of $d$}
	\soln{Solution}
	
	\item \qn{Use composition to express $s$ as a function of $t$}
	\soln{Solution}
\end{enumerate}

\section*{Problem 63}

\qn{\textbf{The Heaviside Function} The Heaviside function $H$ is defined by $$H(t)=\begin{cases} 0 \quad\text{if }t<0 \\ 1 \quad\text{if }t \ge 0\end{cases}$$ It is used in the study of electric circuits to represent the sudden surge of electric current, or voltage, when a switch is
instantaneously turned on.}

\begin{enumerate}
	\item \qn{Sketch the graph of the Heaviside function}
	\soln{Solution}
	
	\item \qn{Sketch the graph of the voltage $V(t)$ in a circuit if the switch is turned on at time $t=0$ and 120 volts are applied instantaneously to the circuit. Write a formula for $V(t)$ in terms of $H(t)$}
	\soln{Solution}
	
	\item \qn{Sketch the graph of the voltage $V(t)$ in a circuit if the switch is turned on at time $t=5$ seconds and 240 volts are applied instantaneously to the circuit. Write a formula for $V(t)$ in terms of $H(t)$. (Note that starting at $t=5$ corresponds to a translation)}
	\soln{Solution}
\end{enumerate}

\section*{Problem 64}

\qn{\textbf{The Ramp Function} The Heaviside function defined in Problem 63 can also be used to define the ramp function $y=ctH(t)$, which represents a gradual increase in voltage or current in a circuit}

\begin{enumerate}
	\item \qn{Sketch the graph of the ramp function $y=tH(t)$}
	\soln{Solution}
	
	\item \qn{Sketch the graph of the voltage $V(t)$ in a circuit if the switch is turned on at time $t=0$ and the voltage is gradually increased to 120 volts over a 60-second time interval. Write a formula for $V(t)$ in terms of $H(t)$ for $t \le 60$}
	\soln{Solution}
	
	\item \qn{Sketch the graph of the voltage $V(t)$ in a circuit if the switch is turned on at time $t=7$ seconds and the voltage is gradually increased to 100 volts over a period of 25 seconds. Write a formula for $V(t)$ in terms of $H(t)$ for $t \le 32$}
	\soln{Solution}
\end{enumerate}

\section*{Problem 65}

\qn{Let $f$ and $g$ be linear functions with equations $f(x)=m_1x+b_1$ and $g(x)=m_2x+b_2$. Is $f \circ g$ also a linear? If so, what is the slope of its graph?}

\soln{Solution}

\section*{Problem 66}

\qn{If you invest $x$ dollars at 4\% interest compounded annually, then the amount $A(x)$ of the investment after one year is $A(x)=1.04x$. Find $A \circ A$, $A \circ A \circ A$, $A \circ A \circ A \circ A$. What doe these compositions represent? Find a formula for the composition of $n$ copies of $A$}

\soln{Solution}

\section*{Problem 67}

\begin{enumerate}
	\item \qn{If $g(x)=2x+1$ and $h(x)=4x^2+4x+7$, find a function $f$ such that $f \circ g=h$. (Think about what operations you would have to perform on the formula for $g$ to end up with the formula for $h$)}
	\soln{Solution}
	
	\item \qn{If $f(x)=x+4$ and $h(x)=3x^2+3x+2$, find a function $g$ such that $f \circ g=h$}
	\soln{Solution}
\end{enumerate}

\section*{Problem 68}

\qn{If $f(x)=x+4$ and $h(x)=4x-1$, find a function $g$ such that $g \circ f=h$}

\soln{Solution}

\section*{Problem 69}

\qn{Suppose $g$ is an even function and let $h=f \circ g$. Is $h$ always an even function?}

\soln{Solution}

\section*{Problem 70}

\qn{Support $g$ is an odd function and let $h=f \circ g$. Is $h$ always an odd function? What if $f$ is odd? What if $f$ is even?}

\soln{Solution}

\section*{Problem 71}

\qn{Let $f(x)$ be a function with domain $\mathbb{R}$}

\begin{enumerate}
	\item \qn{Show that $E(x)=f(x)+f(-x)$ is an even function}
	\soln{Solution}
	
	\item \qn{Show that $O(x)=f(x)-f(-x)$ is an odd function}
	\soln{Solution}
	
	\item \qn{Prove that every function $f(x)$ can be written as a fum of an even function and an odd function}
	\soln{Solution}
	
	\item \qn{Express the function $f(x)=2^x+(x-3)^2$ as a sum of an even function and an odd function}
	\soln{Solution}
\end{enumerate}
	
\end{document}



