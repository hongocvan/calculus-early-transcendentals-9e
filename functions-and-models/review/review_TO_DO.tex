\documentclass[11pt]{article}
\usepackage[utf8]{vietnam}

\usepackage[margin=1in]{geometry}
\usepackage{amsmath,amsfonts,amssymb}
\usepackage[none]{hyphenat}
\usepackage{fancyhdr}

\usepackage{multicol}
\usepackage{graphicx}
\usepackage{pgfplots}
\usepackage{wrapfig}
\usepackage{gensymb}
\usepackage{float}

\pagestyle{fancy}
\fancyhead{}
\fancyfoot{}
\fancyhead[L]{\slshape \MakeUppercase{1.6 - Review}}
\fancyhead[R]{\slshape Ho Ngoc Van}
\fancyfoot[C]{\thepage}
\renewcommand{\footrulewidth}{0pt}

\newcommand{\soln}{\subsection*}
\newcommand{\qn}{\textit}
\newcommand{\imagesource}[1]{{\footnotesize Source: #1}}
\newcommand{\imgqn}[1]{
	\begin{figure}[H]
		\centering
		\includegraphics[width=0.35\linewidth]{figs/#1.png}\\
		\imagesource{James Stewart, Calculus: Early Transcendentals [9e]}
	\end{figure}
}
\newcommand{\imgsoln}[1]{
	\begin{figure}[H]
		\centering
		\includegraphics[width=0.5\linewidth]{figs/#1.png}
	\end{figure}
}

\newcommand{\eqtext}[1]{\quad\text{#1}\quad}

\begin{document}

\section*{Problem 1}

\begin{enumerate}
	\item \qn{What is a function? What are its domain and range?}
	\soln{Solution}
	
	\item \qn{What is the graph of a function?}
	\soln{Solution}
	
	\item \qn{How can you tell whether a given curve is the graph of a function?}
	\soln{Solution}
\end{enumerate}

\section*{Problem 2}

\qn{Discuss four ways of representing a function. Illustrate your discussion with examples.}
\soln{Solution}

\section*{Problem 3}

\begin{enumerate}
	\item \qn{What is an even function? How can you tell if a function is even by looking at its graph? Give three examples of an even function.}
	\soln{Solution}
	
	\item \qn{What is an odd function? How can you tell if a function is odd by looking at its graph? Give three examples of an odd function.}
	\soln{Solution}
\end{enumerate}

\section*{Problem 4}

\qn{What is an increasing function?}
\soln{Solution}

\section*{Problem 5}

\qn{What is a mathematical model?}
\soln{Solution}

\section*{Problem 6}

\qn{Give an example of each type of function}
\begin{enumerate}
	\item \qn{Linear function}
	\soln{Solution}
	
	\item \qn{Power function}
	\soln{Solution}
	
	\item \qn{Exponential function}
	\soln{Solution}
	
	\item \qn{Quadratic function}
	\soln{Solution}
	
	\item \qn{Polynomial of degree 5}
	\soln{Solution}
	
	\item \qn{Rational function}
	\soln{Solution}
\end{enumerate}

\section*{Problem 7}

\qn{Sketch by hand, on the same axes, the graphs of the following functions}
\begin{enumerate}
	\item \qn{$f(x)=x$}
	\soln{Solution}
	
	\item \qn{$g(x)=x^2$}
	\soln{Solution}
	
	\item \qn{$h(x)=x^3$}
	\soln{Solution}
	
	\item \qn{$j(x)=x^4$}
	\soln{Solution}
\end{enumerate}

\section*{Problem 8}

\qn{Draw, by hand, rough sketch of the graph of each function}
\begin{enumerate}
	\item \qn{$y=\sin{x}$}
	\soln{Solution}
	
	\item \qn{$y=\tan{x}$}
	\soln{Solution}
	
	\item \qn{$y=e^x$}
	\soln{Solution}
	
	\item \qn{$y=\ln{x}$}
	\soln{Solution}
	
	\item \qn{$y=1/x$}
	\soln{Solution}
	
	\item \qn{$y=|x|$}
	\soln{Solution}
	
	\item \qn{$y=\sqrt{x}$}
	\soln{Solution}
	
	\item \qn{$y=\tan^{-1}{x}$}
	\soln{Solution}
\end{enumerate}

\section*{Problem 9}

\qn{Suppose that $f$ has domain $A$ and $g$ has domain $B$}
\begin{enumerate}
	\item \qn{What is the domain of $f+g$?}
	\soln{Solution}
	
	\item \qn{What is the domain of $fg$?}
	\soln{Solution}
	
	\item \qn{What is the domain of $f/g$?}
	\soln{Solution}
\end{enumerate}

\section*{Problem 10}

\qn{How is the composite function $f \circ g$ defined? What is its domain?}
\soln{Solution}

\section*{Problem 11}

\qn{Suppose the graph of $f$ is given. Write an equation for each of the graphs that are obtained from the graph of $f$ as follows.}
\begin{enumerate}
	\item \qn{Shift 2 units upward}
	\soln{Solution}
	
	\item \qn{Shift 2 units downward}
	\soln{Solution}
	
	\item \qn{Shift 2 units to the right}
	\soln{Solution}
	
	\item \qn{Shift 2 units to the left}
	\soln{Solution}
	
	\item \qn{Reflect about the $x$-axis}
	\soln{Solution}
	
	\item \qn{Reflect about the $y$-axis}
	\soln{Solution}
	
	\item \qn{Stretch vertically by a factor of 2}
	\soln{Solution}
	
	\item \qn{Shrink vertically by a factor of 2}
	\soln{Solution}
	
	\item \qn{Stretch horizontally by a factor of 2}
	\soln{Solution}
	
	\item \qn{Shrink horizontally by a factor of 2}
	\soln{Solution}
\end{enumerate}

\section*{Problem 12}

\begin{enumerate}
	\item \qn{What is a one-to-one function? How can you tell if a function is one-to-one by looking at its graph?}
	\soln{Solution}
	
	\item \qn{If $f$ is a one-to-one function, how is its inverse function $f^{-1}$ defined? How do you obtain the graph of $f^{-1}$ from the graph of $f$?}
	\soln{Solution}
\end{enumerate}

\section*{Problem 13}

\begin{enumerate}
	\item \qn{How is the inverse sine function $f(x)=\sin^{-1}x$ defined? What are its domain and range?}
	\soln{Solution}
	
	\item \qn{How is the inverse cosine function $f(x)=\cos^{-1}x$ defined? What are its domain and range?}
	\soln{Solution}
	
	\item \qn{How is the inverse tangent function $f(x)=\tan^{-1}x$ defined? What are its domain and range?}
	\soln{Solution}
\end{enumerate}

\paragraph{Determine whether the statement is true or false. If it is true, explain why. If it is false, explain why or give an example that disproves the statement.}

\section*{Problem 14}

\qn{If $f$ is a function, then $f(s+t)=f(s)+f(t)$}
\soln{Solution}

\section*{Problem 15}

\qn{If $f(s)=f(t)$, then $s=t$}
\soln{Solution}

\section*{Problem 16}

\qn{If $f$ is a function, then $f(3x)=3f(x)$}
\soln{Solution}

\section*{Problem 17}

\qn{If the function $f$ has an inverse and $f(2)=3$, then $f^{-1}(3)=2$}
\soln{Solution}

\section*{Problem 18}

\qn{A vertical line intersects the graph of a function at most once}
\soln{Solution}

\section*{Problem 19}

\qn{If $f$ and $g$ are functions, then $f \circ g = g \circ f$}
\soln{Solution}

\section*{Problem 20}

\qn{If $f$ is one-to-one, then $f^{-1}(x)=\frac{1}{f(x)}$}
\soln{Solution}

\section*{Problem 21}

\qn{You can always divide by $e^x$}
\soln{Solution}

\section*{Problem 22}

\qn{If $0<a<b$, then $\ln{a}<\ln{b}$}
\soln{Solution}

\section*{Problem 23}

\qn{If $x>0$, then $(\ln{x})^6=6\ln{x}$}
\soln{Solution}

\section*{Problem 24}

\qn{If $x>0$ and $a>1$, then $\frac{\ln{x}}{\ln{a}}=\ln(\frac{x}{a})$}
\soln{Solution}

\section*{Problem 25}

\qn{$\tan^{-1}(-1)=3\pi/4$}
\soln{Solution}

\section*{Problem 26}

\qn{$\tan^{-1}{x}=\frac{\sin^{-1}x}{\cos^{-1}x}$}
\soln{Solution}

\section*{Problem 27}

\qn{If $x$ is any real number, then $\sqrt{x^2}=x$}
\soln{Solution}

\section*{Problem 28}

\qn{Let $f$ be the function whose graph is given}
\imgqn{review-28}
\begin{enumerate}
	\item \qn{Estimate the value of $f(2)$}
	\soln{Solution}
	
	\item \qn{Estimate the values of $x$ such that $f(x)=3$}
	\soln{Solution}
	
	\item \qn{State the domain of $f$}
	\soln{Solution}
	
	\item \qn{State the range of $f$}
	\soln{Solution}
	
	\item \qn{On what interval is $f$ increasing?}
	\soln{Solution}
	
	\item \qn{Is $f$ one-to-one? Explain}
	\soln{Solution}
	
	\item \qn{If $f$ even, odd, or neither even nor odd? Explain}
	\soln{Solution}
\end{enumerate}

\section*{Problem 29}

\qn{The graph of $g$ is given}
\imgqn{review-29}
\begin{enumerate}
	\item \qn{State the value of $g(2)$}
	\soln{Solution}
	
	\item \qn{Why is $g$ one-to-one?}
	\soln{Solution}
	
	\item \qn{Estimate the value of $g^{-1}(2)$}
	\soln{Solution}
	
	\item \qn{Estimate the domain of $g^{-1}$}
	\soln{Solution}
	
	\item \qn{Sketch the graph of $g^{-1}$}
	\soln{Solution}
\end{enumerate}

\section*{Problem 30}

\qn{If $f(x)=x^2-2x+3$, evaluate the difference quotient $$\frac{f(a+h)-f(a)}{h}$$}
\soln{Solution}

\section*{Problem 31}

\qn{Sketch a rough graph of the yield of a crop as a function the amount of fertilizer used}
\soln{Solution}

\section*{Problem 32}

\qn{Find the domain and range of the function. Write your answer in interval notation $$f(x)=2/(3x-1)$$}
\soln{Solution}

\section*{Problem 33}

\qn{Find the domain and range of the function. Write your answer in interval notation $$g(x)=\sqrt{16-x^4}$$}
\soln{Solution}

\section*{Problem 34}

\qn{Find the domain and range of the function. Write your answer in interval notation $$h(x)=\ln(x+6)$$}
\soln{Solution}

\section*{Problem 35}

\qn{Find the domain and range of the function. Write your answer in interval notation $$F(t)=3+\cos(2t)$$}
\soln{Solution}

\section*{Problem 36}

\qn{Suppose that the graph of $f$ is given. Describe how the graphs of the following functions can be obtained from the graph of}
\begin{enumerate}
	\item \qn{$y=f(x)+5$}
	\soln{Solution}
	
	\item \qn{$y=f(x+5)$}
	\soln{Solution}
	
	\item \qn{$y=1+2f(x)$}
	\soln{Solution}
	
	\item \qn{$y=f(x-2)-2$}
	\soln{Solution}
	
	\item \qn{$y=-f(x)$}
	\soln{Solution}
	
	\item \qn{$y=f^{-1}(x)$}
	\soln{Solution}
\end{enumerate}

\section*{Problem 37}

\qn{The graph of $f$ is given. Draw the graphs of the following functions}
\imgqn{review-37}
\begin{enumerate}
	\item \qn{$y=f(x-8)$}
	\soln{Solution}
	
	\item \qn{$y=-f(x)$}
	\soln{Solution}
	
	\item \qn{$y=2-f(x)$}
	\soln{Solution}
	
	\item \qn{$y=\frac{1}{2}f(x)-1$}
	\soln{Solution}
	
	\item \qn{$y=f^{-1}(x)$}
	\soln{Solution}
	
	\item \qn{$y=f^{-1}(x+3)$}
	\soln{Solution}
\end{enumerate}

\section*{Problem 38}

\qn{Use transformations to sketch the graph of the function $$f(x)=x^3+2$$}
\soln{Solution}

\section*{Problem 39}

\qn{Use transformations to sketch the graph of the function $$f(x)=(x-3)^2$$}
\soln{Solution}

\section*{Problem 40}

\qn{Use transformations to sketch the graph of the function $$y=\sqrt{x+2}$$}
\soln{Solution}

\section*{Problem 41}

\qn{Use transformations to sketch the graph of the function $$y=\ln(x+5)$$}
\soln{Solution}

\section*{Problem 42}

\qn{Use transformations to sketch the graph of the function $$g(x)=1+\cos(2x)$$}
\soln{Solution}

\section*{Problem 43}

\qn{Use transformations to sketch the graph of the function $$h(x)=-e^x+2$$}
\soln{Solution}

\section*{Problem 44}

\qn{Use transformations to sketch the graph of the function $$s(x)=1+0.5^x$$}
\soln{Solution}

\section*{Problem 45}

\qn{Use transformations to sketch the graph of the function $$f(x)=\begin{cases} -x \eqtext{if $x<0$} \\ e^x-1 \eqtext{if $x \ge 0$} \end{cases}$$}
\soln{Solution}

\section*{Problem 46}

\qn{Determine whether $f$ is even, odd, or neither even nor odd}
\begin{enumerate}
	\item \qn{$f(x)=2x^5-3x^2+2$}
	\soln{Solution}
	
	\item \qn{$f(x)=x^3-x^7$}
	\soln{Solution}
	
	\item \qn{$f(x)=e^{-x^2}$}
	\soln{Solution}
	
	\item \qn{$f(x)=1+\sin{x}$}
	\soln{Solution}
	
	\item \qn{$f(x)=1-\cos(2x)$}
	\soln{Solution}
	
	\item \qn{$f(x)=(x+1)^2$}
	\soln{Solution}
\end{enumerate}

\section*{Problem 47}

\qn{Find an expression for the function whose graph consists of the line segment from the point $(-2, 2)$ to the point $(-1, 0)$ together with the top half of the circle with center the origin and radius 1}
\soln{Solution}

\section*{Problem 48}

\qn{If $f(x)=\ln{x}$ and $g(x)=x^2-9$, find the functions, and their domains}
\begin{enumerate}
	\item \qn{$f \circ g$}
	\soln{Solution}
	
	\item \qn{$g \circ f$}
	\soln{Solution}
	
	\item \qn{$f \circ f$}
	\soln{Solution}
	
	\item \qn{$g \circ g$}
	\soln{Solution}
\end{enumerate}

\section*{Problem 49}

\qn{Express the function $F(x)=1/\sqrt{x+\sqrt{x}}$ as a composition of three functions}
\soln{Solution}

\section*{Problem 50}

\qn{Life expectancy has improved dramatically in recent decades. The table gives the life expectancy at birth (in years) of males born in the United States. Use a scatter plot to choose an appropriate type of model. Use your model to predict the life span of a male born in the year 2030}
\imgqn{review-50}
\soln{Solution}

\section*{Problem 51}

\qn{A small-appliance manufacturer finds that it costs \$9000 to produce 1000 toaster ovens a week and \$12,000 to produce 1500 toaster ovens a week}
\begin{enumerate}
	\item \qn{Express the cost as a function of the number of toaster ovens produced, assuming that it is linear. Then sketch the graph.}
	\soln{Solution}
	
	\item \qn{What is the slope of the graph and what does it represent?}
	\soln{Solution}
	
	\item \qn{What is the $y$-intercept of the graph and what does it represent?}
	\soln{Solution}
\end{enumerate}

\section*{Problem 52}

\qn{If $f(x)=2x+4^x$, find $f^{-1}(6)$}
\soln{Solution}

\section*{Problem 53}

\qn{Find the inverse function of $$f(x)=\frac{2x+3}{1-5x}$$}
\soln{Solution}

\section*{Problem 54}

\qn{Use the laws of logarithms to expand each expression}
\begin{enumerate}
	\item \qn{$\ln(x\sqrt{x+1})$}
	\soln{Solution}
	
	\item \qn{$\log_2(\sqrt{\frac{x^2+1}{x-1}})$}
	\soln{Solution}
\end{enumerate}

\section*{Problem 55}

\qn{Express as a single logarithm}
\begin{enumerate}
	\item \qn{$\frac{1}{2}\ln{x}-2\ln(x^2+1)$}
	\soln{Solution}
	
	\item \qn{$\ln(x-3)+\ln(x+3)-2\ln(x^2-9)$}
	\soln{Solution}
\end{enumerate}

\section*{Problem 56}

\qn{Find the exact value of each expression}
\begin{enumerate}
	\item \qn{$e^{2\ln5}$}
	\soln{Solution}
	
	\item \qn{$\log_6(4)+\log_6(54)$}
	\soln{Solution}
	
	\item \qn{$\tan(\arcsin(4/5))$}
	\soln{Solution}
\end{enumerate}

\section*{Problem 57}

\qn{Find the exact value of each expression}
\begin{enumerate}
	\item \qn{$\ln(\frac{1}{e^3})$}
	\soln{Solution}
	
	\item \qn{$\sin(\tan^{-1}1)$}
	\soln{Solution}
	
	\item \qn{$10^{-3\log4}$}
	\soln{Solution}
\end{enumerate}

\section*{Problem 58}

\qn{Solve the equation for $x$. Give both an exact value and a decimal approximation, correct to three decimal places $$e^{2x}=3$$}
\soln{Solution}

\section*{Problem 59}

\qn{Solve the equation for $x$. Give both an exact value and a decimal approximation, correct to three decimal places $$\ln(x^2)=5$$}
\soln{Solution}

\section*{Problem 60}

\qn{Solve the equation for $x$. Give both an exact value and a decimal approximation, correct to three decimal places $$e^{e^x}=10$$}
\soln{Solution}

\section*{Problem 61}

\qn{Solve the equation for $x$. Give both an exact value and a decimal approximation, correct to three decimal places $$\cos^{-1}(x)=2$$}
\soln{Solution}

\section*{Problem 62}

\qn{Solve the equation for $x$. Give both an exact value and a decimal approximation, correct to three decimal places $$\tan^{-1}(ex^2)=\pi/4$$}
\soln{Solution}

\section*{Problem 63}

\qn{Solve the equation for $x$. Give both an exact value and a decimal approximation, correct to three decimal places $$\ln{x}-1=\ln(5+x)-4$$}
\soln{Solution}

\section*{Problem 64}

\qn{The viral load for an HIV patient is 52.0 RNA copies/mL before treatment begins. Eight days later the viral load is half of the initial amount}
\begin{enumerate}
	\item \qn{Find the viral load after 24 days}
	\soln{Solution}
	
	\item \qn{Find the viral load $V(t)$ that remains after $t$ days}
	\soln{Solution}
	
	\item \qn{Find a formula for the inverse of the function $V$ and explain its meaning}
	\soln{Solution}
	
	\item \qn{After how many days will the viral load be reduced to 2.0 RNA copies/mL?}
	\soln{Solution}
\end{enumerate}

\section*{Problem 65}

\qn{The population of a certain species in a limited environment with initial population 100 and carrying capacity 1000 is $$P(t)=\frac{100,000}{100+900e^{-1}}$$ where $t$ is measured in years}
\begin{enumerate}
	\item \qn{Graph this function and estimate how long it takes for the population to reach 900.}
	\soln{Solution}
	
	\item \qn{Find the inverse of this function and explain its meaning.}
	\soln{Solution}
	
	\item \qn{Use the inverse function to find the time required for the population to reach 900. Compare with the result of part (1)}
	\soln{Solution}
\end{enumerate}

\end{document}



